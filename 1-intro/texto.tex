\chapter{Introdução}

%=====================================================

% A introdução geral do documento pode ser apresentada através das seguintes seções: Desafio, Motivação, Proposta, Contribuição e Organização do documento (especificando o que será tratado em cada um dos capítulos). O Capítulo 1 não contém subseções\footnote{Ver o Capítulo \ref{cap-exemplos} para comentários e exemplos de subseções.}.

%=====================================================

Sarcasmo é uma forma de expressão frequentemente utilizada na comunicação humana, caracterizada por 
uma discrepância entre o significado literal das palavras e a intenção real do falante. Sendo de 
difícil detecção até mesmo para humanos, o sarcasmo apresenta um desafio significativo para sistemas 
de processamento de linguagem natural (PLN). A detecção automática de sarcasmo é crucial para
melhorar a compreensão de textos em diversas aplicações, por exemplo, em análises de sentimento. 

Nos últimos anos, a detecção de sarcasmo tem sido abordada com o uso de técnicas de aprendizado de 
máquina, incluindo modelos baseados em redes neurais profundas como LSTM (Long Short-Term Memory) \cite{kumar20}, 
CNN (Convolutional Neural Networks) \cite{razali21} e, mais recentemente, modelos pré-treinados como BERT 
( Bidirectional Encoder Representations from Transformers) \cite{shu24}. Estes modelos têm demonstrado um desempenho 
promissor na identificação de sarcasmo em textos, aproveitando grandes volumes de dados para capturar
padrões linguísticos complexos.

Trabalhos anteriores na área têm explorado diversas abordagens para a detecção de sarcasmo e para 
diversas línguas, incluindo inglês \cite{scola21}, árabe \cite{rahma23} e hindi \cite{swami18}. No entanto, a detecção de sarcasmo em português ainda 
é um campo relativamente inexplorado, com poucos estudos focados em textos brasileiros. De fato, não foram
encontrados muitos trabalhos abordando especificamente a detecção de sarcasmo em português, o que destaca a 
necessidade de pesquisa adicional nesta área.

Este trabalho propõe uma abordagem para a detecção automática de sarcasmo em textos em português,
utilizando modelos de aprendizado profundo, com ênfase em BERT e LSTM. A contribuição principal deste 
trabalho é o desenvolvimento e avaliação de modelos eficazes para a detecção de sarcasmo em português, 
além da criação de um conjunto de dados anotados especificamente para este propósito.

A base de dados utilizada neste trabalho é composta por manchetes de notícias sarcásticas, coletadas de
fontes online: como o site The Onion, conhecido por seu conteúdo satírico. Esta base já tinha sido 
utilizada em trabalhos anteriores, como o de \citet{nayak22}. A escolha desta base de dados se justifica 
primeiramente por estar disponível publicamente, além de conter textos gramáticalmente corretos e bem 
estruturados, o que facilita a análise e o desenvolvimento de modelos de detecção de sarcasmo.

Muitos trabalhos anteriores focaram em textos curtos, como \textit{tweets}, que frequentemente contêm erros 
gramaticais, abreviações, linguagem coloquials, \textit{hashtags}, \textit{emojis} e uma variedade de outros elementos 
que podem complicar a análise linguística. Muitos trabalhos tem que empregar técnicas adicionais de
pré-processamento para lidar com essas características, o que pode introduzir ruído e afetar negativamente o 
desempenho dos modelos. Além disso, as bases do twitter, ou de outras redes sociais, foram coletadas 
utilizando palavras-chave ou \textit{hashtags} específicas. No twitter, por exemplo, usuários podem marcar suas 
postagens como sarcásticas usando \textit{hashtags} como \#sarcasm ou \#irony. No entanto, nem todos os usuários
utilizam essas hashtags de forma consistente, o que pode levar a uma representação incompleta do sarcasmo
na base de dados

Em contraste, as manchetes de notícias sarcásticas são geralmente escritas por profissionais e
passam por um processo editorial, resultando em textos mais claros e coerentes. Isso pode facilitar a 
detecção de sarcasmo, uma vez que os modelos podem se concentrar nas nuances linguísticas sem a interferência de 
ruídos comuns em textos de redes sociais. Portanto, a escolha de uma base de dados composta por 
manchetes de notícias sarcásticas permite explorar a detecção de sarcasmo em um contexto mais formal e estruturado, 
oferecendo insights valiosos para o desenvolvimento de modelos de PLN eficazes.

A base de dados, porém, apresenta limitações, como o fato de ser composta exclusivamente por textos em inglês. 
Portanto, uma das contribuições deste trabalho é a tradução e adaptação desta base de dados para o português, 
permitindo a avaliação dos modelos propostos em um contexto linguístico diferente. 
%Como a tradução automática foi feita será explicado no Capítulo \ref{chap:modelo_proposto}. 

%\section{Considerações}

A organização deste documento é a seguinte: o Capítulo 2 apresenta a fundamentação teórica sobre os 
principais conceitos e técnicas relacionados à detecção de sarcasmo e aos modelos de aprendizado profundo
utilizados. O Capítulo 3 explora trabalhos relacionados na área de detecção de sarcasmo, destacando as abordagens 
e resultados obtidos. O Capítulo 4 detalha o modelo proposto, incluindo a arquitetura, os dados utilizados e o 
processo de treinamento. O Capítulo 5 apresenta os experimentos realizados, os resultados obtidos e a análise 
dos mesmos. Finalmente, o Capítulo 6 conclui o documento, resumindo as principais contribuições e sugerindo
direções para trabalhos futuros.