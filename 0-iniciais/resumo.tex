\begin{resumo}

O sarcasmo, presente no cotidiano, é frequentemente empregado para criticar, muitas vezes com humor, 
pessoas, instituições ou eventos. Por depender de contexto, conhecimento de mundo, tom e implicaturas 
pragmáticas, pode ser difícil de compreender até mesmo para seres humanos, já que o sentido pretendido 
geralmente contraria o literal. Consequentemente, a detecção automática de sarcasmo constitui um 
desafio relevante em Processamento de Linguagem Natural (PLN), pois erros na sua identificação podem 
distorcer tarefas a jusante, como análise de sentimento e mineração de opiniões. Nos últimos anos, o 
tema ganhou destaque justamente por seu impacto na qualidade dessas aplicações e pela necessidade de 
modelos capazes de captar nuances semânticas e contextuais. Neste trabalho, foram abordados dois 
modelos para a detecção de sarcasmo. O primeiro modelo é baseado em 
BiLSTM (Bidirectional Long Short-Term Memory), já o segundo modelo é baseado em 
BERT (Bidirectional Encoder Representations from Transformers), juntamente com BiLSTM. Como conjunto de 
dados, foram utilizadas manchetes sarcásticas retiradas do site The Onion e traduzidas para o 
português utilizando a API do ChatGPT. Ao final do trabalho, é feita uma comparação entre os dois 
modelos. Conclui-se que o modelo BERT com BiLSTM foi superior, alcançando F1 de 0.85.

\end{resumo}
